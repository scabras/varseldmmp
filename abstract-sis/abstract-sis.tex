%%%%%%%%%%%%%%%%%%%% author.tex %%%%%%%%%%%%%%%%%%%%%%%%%%%%%%%%%%%
%
% sample root file for your "contribution" to a contributed volume
%
% Use this file as a template for your own input.
%
%%%%%%%%%%%%%%%% Springer %%%%%%%%%%%%%%%%%%%%%%%%%%%%%%%%%%


% RECOMMENDED %%%%%%%%%%%%%%%%%%%%%%%%%%%%%%%%%%%%%%%%%%%%%%%%%%%
\documentclass[graybox]{svmult}

% choose options for [] as required from the list
% in the Reference Guide

\usepackage{mathptmx}       % selects Times Roman as basic font
\usepackage{helvet}         % selects Helvetica as sans-serif font
\usepackage{courier}        % selects Courier as typewriter font
\usepackage{type1cm}        % activate if the above 3 fonts are
                            % not available on your system
%
\usepackage{makeidx}         % allows index generation
\usepackage{graphicx}        % standard LaTeX graphics tool
                             % when including figure files
\usepackage{multicol}        % used for the two-column index
\usepackage[bottom]{footmisc}% places footnotes at page bottom

% see the list of further useful packages
% in the Reference Guide

\makeindex             % used for the subject index
                       % please use the style svind.ist with
                       % your makeindex program

%%%%%%%%%%%%%%%%%%%%%%%%%%%%%%%%%%%%%%%%%%%%%%%%%%%%%%%%%%%%%%%%%%%%%%%%%%%%%%%%%%%%%%%%%

\begin{document}

\title*{Non-parametric prior for variables selection in linear regression}
\titlerunning{NP prior for variable selection}
\subtitle{\emph{Uso di una a priori non parametrica per selezione di variabili nella regressione lineare}}
\author{Stefano Cabras}
\authorrunning{S. Cabras et al.}
\institute{Stefano Cabras \at Dipartimento di Matematica e Informatica, Università degli Studi di Cagliari, Via Ospedale, 72 - 09124 Cagliari (Italy) and Department of Statistics, Universidad Carlos III de Madrid, C/Madrid, 126 - 09042 Getafe (Spain), \email{s.cabras@unica.it - stefano.cabras@uc3m.es}
\and Name of Second Author \at Name, Address of Institute \email{name@email.address}}
%
% Use the package "url.sty" to avoid
% problems with special characters
% used in your e-mail or web address
%
\maketitle

\abstract{We consider the problem of variable selection in ordinary linear regression model where the number of covariates is larger than the sample size and big enough to prevent exploration of all possible models. In this context, a stochastic model exploration with Gibbs-sampling is usually performed and limited to models of size no larger than the sample size. Gibbs-sampling also incorporates evidence of Bayes Factors derived either from conventional priors or from non-local priors. The focus here is on the prior on model space for which we explore the use of the Dirichlet Process (DP) prior on the space of all possible models. Models posterior probabilities are approximated combining the DP prior and the output of the Gibbs-sampling.}
\abstract{\emph{Abstract in Italian}}
\keywords{Conventional priors, Dirichlet Process, Non-local priors, Ordinary linear regression, Variable selection}

\section{Introduction}
\label{sec:intro}
\cite{BVSNLP}

\section{Dirichlet process prior on model space}
\label{sec:dp}

....


\subsection{Variational algorithm for approximate model posterior probability}
\label{subsec:variational}
....


\section{Examples}
\label{sec:ex}


....

\section{Remarks}
\label{sec:remarks}



\bibliographystyle{spbasic.bst}
\bibliography{biblio-msdpmp.bib}

\end{document}




Instead of simply listing headings of different levels we recommend to
let every heading be followed by at least a short passage of text.
Further on please use the \LaTeX\ automatism for all your
cross-references\index{cross-references} and citations\index{citations}
as has already been described in Sect.~\ref{sec:2}.

\begin{quotation}
Please do not use quotation marks when quoting texts! Simply use the \verb|quotation| environment -- it will automatically render Springer's preferred layout.
\end{quotation}


\subsubsection{Subsubsection Heading}
Instead of simply listing headings of different levels we recommend to
let every heading be followed by at least a short passage of text.
Further on please use the \LaTeX\ automatism for all your
cross-references and citations as has already been described in
Sect.~\ref{subsec:2}, see also Fig.~\ref{fig:1}\footnote{If you copy
text passages, figures, or tables from other works, you must obtain
\textit{permission} from the copyright holder (usually the original
publisher). Please enclose the signed permission with the manucript. The
sources\index{permission to print} must be acknowledged either in the
captions, as footnotes or in a separate section of the book.}

Please note that the first line of text that follows a heading is not indented, whereas the first lines of all subsequent paragraphs are.

% For figures use
%
\begin{figure}[b]
\sidecaption
% Use the relevant command for your figure-insertion program
% to insert the figure file.
% For example, with the graphicx style use
\includegraphics[scale=.65]{figure}
%
% If no graphics program available, insert a blank space i.e. use
%\picplace{5cm}{2cm} % Give the correct figure height and width in cm
%
\caption{If the width of the figure is less than 7.8 cm use the \texttt{sidecapion} command to flush the caption on the left side of the page. If the figure is positioned at the top of the page, align the sidecaption with the top of the figure -- to achieve this you simply need to use the optional argument \texttt{[t]} with the \texttt{sidecaption} command}
\label{fig:1}       % Give a unique label
\end{figure}


\paragraph{Paragraph Heading} %
Instead of simply listing headings of different levels we recommend to
let every heading be followed by at least a short passage of text.
Further on please use the \LaTeX\ automatism for all your
cross-references and citations as has already been described in
Sect.~\ref{sec:2}.

Please note that the first line of text that follows a heading is not indented, whereas the first lines of all subsequent paragraphs are.

For typesetting numbered lists we recommend to use the \verb|enumerate| environment -- it will automatically render Springer's preferred layout.

\begin{enumerate}
\item{Livelihood and survival mobility are oftentimes coutcomes of uneven socioeconomic development.}
\begin{enumerate}
\item{Livelihood and survival mobility are oftentimes coutcomes of uneven socioeconomic development.}
\item{Livelihood and survival mobility are oftentimes coutcomes of uneven socioeconomic development.}
\end{enumerate}
\item{Livelihood and survival mobility are oftentimes coutcomes of uneven socioeconomic development.}
\end{enumerate}


\subparagraph{Subparagraph Heading} In order to avoid simply listing headings of different levels we recommend to let every heading be followed by at least a short passage of text. Use the \LaTeX\ automatism for all your cross-references and citations as has already been described in Sect.~\ref{sec:2}, see also Fig.~\ref{fig:2}.

For unnumbered list we recommend to use the \verb|itemize| environment -- it will automatically render Springer's preferred layout.

\begin{itemize}
\item{Livelihood and survival mobility are oftentimes coutcomes of uneven socioeconomic development, cf. Table~\ref{tab:1}.}
\begin{itemize}
\item{Livelihood and survival mobility are oftentimes coutcomes of uneven socioeconomic development.}
\item{Livelihood and survival mobility are oftentimes coutcomes of uneven socioeconomic development.}
\end{itemize}
\item{Livelihood and survival mobility are oftentimes coutcomes of uneven socioeconomic development.}
\end{itemize}

\begin{figure}[t]
\sidecaption[t]
% Use the relevant command for your figure-insertion program
% to insert the figure file.
% For example, with the option graphics use
\includegraphics[scale=.65]{figure}
%
% If no graphics program available, insert a blank space i.e. use
%\picplace{5cm}{2cm} % Give the correct figure height and width in cm
%
%\caption{Please write your figure caption here}
\caption{If the width of the figure is less than 7.8 cm use the \texttt{sidecapion} command to flush the caption on the left side of the page. If the figure is positioned at the top of the page, align the sidecaption with the top of the figure -- to achieve this you simply need to use the optional argument \texttt{[t]} with the \texttt{sidecaption} command}
\label{fig:2}       % Give a unique label
\end{figure}

\runinhead{Run-in Heading Boldface Version} Use the \LaTeX\ automatism for all your cross-references and citations as has already been described in Sect.~\ref{sec:2}.

\subruninhead{Run-in Heading Italic Version} Use the \LaTeX\ automatism for all your cross-refer\-ences and citations as has already been described in Sect.~\ref{sec:2}\index{paragraph}.
% Use the \index{} command to code your index words
%
% For tables use
%
\begin{table}
\caption{Please write your table caption here}
\label{tab:1}       % Give a unique label
%
% Follow this input for your own table layout
%
\begin{tabular}{p{2cm}p{2.4cm}p{2cm}p{4.9cm}}
\hline\noalign{\smallskip}
Classes & Subclass & Length & Action Mechanism  \\
\noalign{\smallskip}\svhline\noalign{\smallskip}
Translation & mRNA$^a$  & 22 (19--25) & Translation repression, mRNA cleavage\\
Translation & mRNA cleavage & 21 & mRNA cleavage\\
Translation & mRNA  & 21--22 & mRNA cleavage\\
Translation & mRNA  & 24--26 & Histone and DNA Modification\\
\noalign{\smallskip}\hline\noalign{\smallskip}
\end{tabular}
$^a$ Table foot note (with superscript)
\end{table}
%
\section{Section Heading}
\label{sec:3}
% Always give a unique label
% and use \ref{<label>} for cross-references
% and \cite{<label>} for bibliographic references
% use \sectionmark{}
% to alter or adjust the section heading in the running head
Instead of simply listing headings of different levels we recommend to
let every heading be followed by at least a short passage of text.
Further on please use the \LaTeX\ automatism for all your
cross-references and citations as has already been described in
Sect.~\ref{sec:2}.

Please note that the first line of text that follows a heading is not indented, whereas the first lines of all subsequent paragraphs are.

If you want to list definitions or the like we recommend to use the Springer-enhanced \verb|description| environment -- it will automatically render Springer's preferred layout.

\begin{description}[Type 1]
\item[Type 1]{That addresses central themes pertainng to migration, health, and disease. In Sect.~\ref{sec:1}, Wilson discusses the role of human migration in infectious disease distributions and patterns.}
\item[Type 2]{That addresses central themes pertainng to migration, health, and disease. In Sect.~\ref{subsec:2}, Wilson discusses the role of human migration in infectious disease distributions and patterns.}
\end{description}

\subsection{Subsection Heading} %
In order to avoid simply listing headings of different levels we recommend to let every heading be followed by at least a short passage of text. Use the \LaTeX\ automatism for all your cross-references and citations citations as has already been described in Sect.~\ref{sec:2}.

Please note that the first line of text that follows a heading is not indented, whereas the first lines of all subsequent paragraphs are.

\begin{svgraybox}
If you want to emphasize complete paragraphs of texts we recommend to use the newly defined Springer class option \verb|graybox| and the newly defined environment \verb|svgraybox|. This will produce a 15 percent screened box 'behind' your text.

If you want to emphasize complete paragraphs of texts we recommend to use the newly defined Springer class option and environment \verb|svgraybox|. This will produce a 15 percent screened box 'behind' your text.
\end{svgraybox}


\subsubsection{Subsubsection Heading}
Instead of simply listing headings of different levels we recommend to
let every heading be followed by at least a short passage of text.
Further on please use the \LaTeX\ automatism for all your
cross-references and citations as has already been described in
Sect.~\ref{sec:2}.

Please note that the first line of text that follows a heading is not indented, whereas the first lines of all subsequent paragraphs are.

\begin{theorem}
Theorem text goes here.
\end{theorem}
%
% or
%
\begin{definition}
Definition text goes here.
\end{definition}

\begin{proof}
%\smartqed
Proof text goes here.
\qed
\end{proof}

\paragraph{Paragraph Heading} %
Instead of simply listing headings of different levels we recommend to
let every heading be followed by at least a short passage of text.
Further on please use the \LaTeX\ automatism for all your
cross-references and citations as has already been described in
Sect.~\ref{sec:2}.

Note that the first line of text that follows a heading is not indented, whereas the first lines of all subsequent paragraphs are.
%
% For built-in environments use
%
\begin{theorem}
Theorem text goes here.
\end{theorem}
%
\begin{definition}
Definition text goes here.
\end{definition}
%
\begin{proof}
\smartqed
Proof text goes here.
\qed
\end{proof}
%
\begin{acknowledgement}
If you want to include acknowledgments of assistance and the like at the end of an individual chapter please use the \verb|acknowledgement| environment -- it will automatically render Springer's preferred layout.
\end{acknowledgement}
%
\section*{Appendix}
\addcontentsline{toc}{section}{Appendix}
%
%
When placed at the end of a chapter or contribution (as opposed to at the end of the book), the numbering of tables, figures, and equations in the appendix section continues on from that in the main text. Hence please \textit{do not} use the \verb|appendix| command when writing an appendix at the end of your chapter or contribution. If there is only one the appendix is designated ``Appendix'', or ``Appendix 1'', or ``Appendix 2'', etc. if there is more than one.

\begin{equation}
a \times b = c
\end{equation}

